\documentclass[times, utf8, seminar]{fer}

\usepackage{booktabs}
\usepackage{hyperref}
\usepackage{enumitem}
\usepackage{mathtools}
\usepackage{listings}
\usepackage{tikz-uml}
\usepackage{multirow}
\usepackage{tikz}
\usepackage{svg}

\begin{document}

\title{Sybil napadi u društvenim mrežama i zaštita od njih}
\author{Antun Razum}
\voditelj{prof. dr. sc. Siniša Srbljić}

\maketitle

\tableofcontents

\chapter{Uvod}

\chapter{Grafovi društvenih mreža}

\chapter{Sybil napadi}

\chapter{Zaštita od sybil napada}

\chapter{Zaključak}

\bibliography{literatura}
\bibliographystyle{fer}

\begin{sazetak}
  Sybil napad je napad kojim se pokušava srušiti sustav reputacije stvaranjem lažnih identiteta u peer-to-peer mrežama koji djeluju na slican način. Društvene mreže su vrlo česta meta sybil napada. Zaštita od sybil napada na društevnim mrežama temelji se na algoritamskim svojstvima grafova društevnih mreža putem kojih se računa razina povjerenja koja se može pridijeliti proizvoljnom čvoru grafa.

  \kljucnerijeci{društvene mreže, sybil napad, teorija grafova, sigurnost podataka, idenditet}
\end{sazetak}

\end{document}
