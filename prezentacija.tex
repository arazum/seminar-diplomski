\documentclass{beamer}
\setbeamertemplate{section in toc}[sections numbered]

\usepackage[utf8]{inputenc}
\usepackage[croatian]{babel}
\usepackage[T1]{fontenc}
\usepackage{lmodern}

\usepackage{enumitem}
\usepackage{mathtools}
\usepackage{listings}
\usepackage{tikz-uml}
\usepackage{multirow}
\usepackage{tikz}
\usepackage{svg}

\usetheme[numbering=fraction]{metropolis}

\title{Sybil napadi u društvenim mrežama i zaštita od njih}
\author{Antun Razum\\Voditelj: prof. dr. sc. Siniša Srbljić}
\institute{Fakultet elektrotehnike i računarstva}
\date{\today}

\begin{document}

\maketitle

\begin{frame}{Sadržaj}
  \tableofcontents
\end{frame}

\section{Uvod}

\begin{frame}{Sybil napadi}
  \begin{enumerate}
    \item napadi na distribuiranim sustavima poput senzorskih i \textit{peer-to-peer} mreža
    \item napadač stvara velik broj lažnih identiteta preko kojih utječe na ponašanje sustava
    \item danas aktualno na društvenim mrežama
  \end{enumerate}
\end{frame}

\begin{frame}{Primjeri napada}
\end{frame}

\section{Drugi pristupi}

\begin{frame}{Središnji autoritet}
\end{frame}

\begin{frame}{Decentralizirani pristupi}
\end{frame}

\begin{frame}{Motivacija}
\end{frame}

\section{Pojmovi i definicije}

\begin{frame}{Model društvene mreže}
\end{frame}

\begin{frame}{Slučajne šetnje}
\end{frame}

\section{Obrana od sybil napada}

\begin{frame}{Pretpostavke algoritma}
\end{frame}

\begin{frame}{Identifikacija sybil čvorova}
\end{frame}

\begin{frame}{Pronalazak sybil grupa}
\end{frame}

\section{Rezultati}

\begin{frame}{Korištene metode i skupovi podataka}
\end{frame}

\begin{frame}{Rezultati}
\end{frame}

\section{Zaključak}

\begin{frame}{Zaključak}
\end{frame}

\begin{frame}[standout]
  \Huge{\centerline{Pitanja?}}
\end{frame}

\begin{frame}[standout]
  \Huge{\centerline{Hvala na pažnji!}}
\end{frame}

\end{document}
